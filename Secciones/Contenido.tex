\section{¿Que es GitHub?} 
GitHub es una plataforma de desarrollo colaborativo de software para alojar proyectos utilizando el sistema de control de versiones Git.
Nota:
El código se almacena de forma pública, aunque también se puede hacer de forma privada, creando una cuenta de pago.

\begin{center}
\includegraphics[width=14cm]{./Imagenes/imagen1} 
\end{center}

\section{¿Para que sirve?} 
GitHub aloja tu repositorio de código y te brinda herramientas muy útiles para el trabajo en equipo, dentro de un proyecto.
Además de eso, puedes contribuir a mejorar el software de los demás. Para poder alcanzar esta meta, GitHub provee de funcionalidades para hacer un fork y solicitar pulls.

\begin{center}
\includegraphics[width=14cm]{./Imagenes/imagen2} 
\end{center}
Además de eso, puedes contribuir a mejorar el software de los demás. Para poder alcanzar esta meta, GitHub provee de funcionalidades para hacer un fork y solicitar pulls.

\section{¿Qué herramientas proporciona?} 
En la actualidad, GitHub es mucho más que un servicio de alojamiento de código. Además de éste, se ofrecen varias herramientas útiles para el trabajo en equipo. Entre ellas, caben destacar:

\begin{center}
\includegraphics[width=14cm]{./Imagenes/imagen3} 
\end{center}

\begin{itemize}
\item Una wiki para el mantenimiento de las distintas versiones de las páginas.
\item Un sistema de seguimiento de problemas que permiten a los miembros de tu equipo detallar un problema con tu software o una sugerencia que deseen hacer.
\item Una herramienta de revisión de código, donde se pueden añadir anotaciones en cualquier punto de un fichero y debatir sobre determinados cambios realizados en un commit específico.
\item Un visor de ramas donde se pueden comparar los progresos realizados en las distintas ramas de nuestro repositorio.
\end{itemize}



\section{¿Qué uso le daremos?} 
En nuestra carrera de Ingenieria de Sistemas, fuimos aprendiendo cosas y creando aplicaciones en diferentes lenguajes, es por eso que presentamos esta gran herramienta enfocada al crecimiento de proyectos comunitarios y libres.

\begin{center}
\includegraphics[width=14cm]{./Imagenes/imagen4} 
\end{center}

En esta página podremos crear una cuenta gratuita y comenzar a subir repositorios de código (o crearlos desde 0), para que con la ayuda de todos ese proyecto mejore; así como también fortalecer los proyectos de los demás para crecer como grupo.

\begin{center}
\includegraphics[width=14cm]{./Imagenes/imagen5} 
\end{center}

\section{CONCLUSIONES} 
\begin{itemize}
\item GitHub (y otras plataformas Web) son las nuevas herramientas en nuestra caja de herramientas de estudiantes.
\item Debemos analizarlas para comprender su papel e impacto.
\item Nuevos retos: integración, propiedad intelectual, nuevas herramientas de gestión...

\end{itemize}





